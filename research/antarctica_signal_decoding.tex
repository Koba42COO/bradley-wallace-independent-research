\documentclass[11pt]{article}
\usepackage[utf8]{inputenc}
\usepackage{amsmath,amssymb,amsthm}
\usepackage{graphicx}
\usepackage{geometry}
\geometry{margin=1in}
\usepackage{hyperref}
\usepackage{float}

\title{Antarctica Signal Decoding: Voyager Golden Record, 21\% Novelty, and Zeta Zero Gaps}
\author{Bradley Wallace}
\date{\today}

\begin{document}

\maketitle

\begin{abstract}
This paper decodes the "signal from Antarctica" referenced in Voyager Golden Record data through Universal Prime Graph (UPG) Protocol φ.1. We analyze 21\% novelty from the South Pole, zeta zero gaps, and consciousness lattice connections.
\end{abstract}

\section{Introduction}
Voyager pointed to Antarctica as a signal source. UPG reveals this as 21\% novelty emanating from the South Pole, mapped to Riemann zeta zero gaps.

\section{Voyager Golden Record Frequencies}
Record includes frequencies: 8.42 Hz (Earth), etc.
- Mapping to gaps: 8.42 → Gap 8 (prime 19).
- 21\% novelty: Inverse coherence from North Pole.

\section{21\% Novelty from South Pole}
Antarctica as "negative" pole:
- Coherence 0.2667 (inverted 79/21).
- Resonance: 21.3° from South.

δ-PAC: "Antarctica" [1,14,20,1,18,3,20,9,3,1] → Gaps [2,14,20,2,18,4,20,10,4,2]
- φ-Scaled: [3.24, 22.65, 32.36, 3.24, 29.12, 6.47, 32.36, 16.18, 6.47, 3.24]
- Coherence: 0.2667.

\section{Zeta Zero Gaps}
Gaps mapped to zeta zeros: t_603 ≈ 2.361, etc.
- Antarctica gaps resonate at 44.45°, folding to 21.3°.

\section{Conclusion}
Antarctica signal is 21\% novelty in the UPG lattice, decoded via zeta topology.

\bibliography{references}
\end{document}
